\documentclass[a4paper,11pt]{report}

\usepackage{geometry}
\geometry{a4paper,left=18mm,right=18mm, top=2cm, bottom=2cm}

\usepackage{lmodern}
\usepackage[T1]{fontenc}
\usepackage{eurosym}
\usepackage{setspace}
\usepackage[utf8]{inputenc}
\usepackage{graphicx}
\usepackage[ngerman]{babel}
\usepackage[solution_off]{srdp-mathematik} % solution_on/off
\setcounter{Zufall}{0}
\usepackage{listings}
\usepackage{fancyhdr}
\renewcommand{\headrulewidth}{0pt}

\pagestyle{plain} %PAGESTYLE: empty, plain, fancy
\onehalfspacing %Zeilenabstand
\setcounter{secnumdepth}{-1} % keine Nummerierung der Ueberschriften


%
%
%%%%%%%%%%%%%%%%%%%%%%%%%%%%%%%%%%%%%%%%%%%%%%%%%%%%%%%%%%%%%%%%%%%%%%%%%%%%%%%%%%%%%%%%%% DOKUMENT - ANFANG %%%%%%%%%%%%%%%%%%%%%%%%%%%%%%%%%%%%%%%%%%%%%%%%%%%%%%%%%%%%%%%%%%%%%%%%%%%%%%%%%%%%%%%
%
%
\begin{document}
\section{GeoGebra-Optimierung}\leer

\begin{beispiel}[FA 3.1]{1} %PUNKTE DES BEISPIELS
Gegeben sind vier Graphen von Potenzfunktionen und sechs Funktionsgleichungen.

Ordne den vier Graphen jeweils die entsprechende Funktionsgleichung (aus A bis F) zu!

\zuordnen{
				title1={Funktionsgleichungen}, 		%Titel Antwortmoeglichkeiten
				A={$f(x)=x^2+1$}, 				%Moeglichkeit A  
				B={$f(x)=x^2-1$}, 				%Moeglichkeit B  
				C={$f(x)=-x^2+1$}, 				%Moeglichkeit C  
				D={$f(x)=x^{-2}+1$}, 				%Moeglichkeit D  
				E={$f(x)=x^{-2}-1$}, 				%Moeglichkeit E  
				F={$f(x)=-x^{-2}$}, 				%Moeglichkeit F  
				title2={Graphen},		%Titel Zuordnung
				R1={\psset{xunit=0.6cm,yunit=0.6cm,algebraic=true,dimen=middle,dotstyle=o,dotsize=5pt 0,linewidth=0.8pt,arrowsize=3pt 2,arrowinset=0.25}
\begin{pspicture*}(-3.7992750623912968,-3.776781160825846)(3.7340282377430136,3.7323769364234036)
\multips(0,-3)(0,1.0){8}{\psline[linestyle=dashed,linecap=1,dash=1.5pt 1.5pt,linewidth=0.4pt,linecolor=gray]{c-c}(-3.7992750623912968,0)(3.7340282377430136,0)}
\multips(-3,0)(1.0,0){8}{\psline[linestyle=dashed,linecap=1,dash=1.5pt 1.5pt,linewidth=0.4pt,linecolor=gray]{c-c}(0,-3.776781160825846)(0,3.7323769364234036)}
\psaxes[labelFontSize=\scriptstyle,xAxis=true,yAxis=true,showorigin=false,Dx=1.,Dy=1.,ticksize=-2pt 0,subticks=0]{->}(0,0)(-3.7992750623912968,-3.776781160825846)(3.7340282377430136,3.7323769364234036)[\scriptsize{$x$},140] [\scriptsize{$f(x)$},-40]
\psplot[linewidth=1.2pt,plotpoints=200]{-3.7992750623912968}{3.7340282377430136}{x^(2.0)-1.0}
\end{pspicture*}},				%1. Antwort rechts
				R2={
\psset{xunit=0.6cm,yunit=0.6cm,algebraic=true,dimen=middle,dotstyle=o,dotsize=5pt 0,linewidth=0.8pt,arrowsize=3pt 2,arrowinset=0.25}
\begin{pspicture*}(-3.7992750623912968,-3.776781160825846)(3.7340282377430136,3.7323769364234036)
\multips(0,-3)(0,1.0){8}{\psline[linestyle=dashed,linecap=1,dash=1.5pt 1.5pt,linewidth=0.4pt,linecolor=gray]{c-c}(-3.7992750623912968,0)(3.7340282377430136,0)}
\multips(-3,0)(1.0,0){8}{\psline[linestyle=dashed,linecap=1,dash=1.5pt 1.5pt,linewidth=0.4pt,linecolor=gray]{c-c}(0,-3.776781160825846)(0,3.7323769364234036)}
\psaxes[labelFontSize=\scriptstyle,xAxis=true,yAxis=true,showorigin=false,Dx=1.,Dy=1.,ticksize=-2pt 0,subticks=0]{->}(0,0)(-3.7992750623912968,-3.776781160825846)(3.7340282377430136,3.7323769364234036)[\scriptsize{$x$},140] [\scriptsize{$f(x)$},-40]
\psplot[linewidth=1.2pt,plotpoints=200]{-3.7992750623912968}{3.7340282377430136}{x^(-2.0)+1.0}
\end{pspicture*}},				%2. Antwort rechts
				R3={\psset{xunit=0.6cm,yunit=0.6cm,algebraic=true,dimen=middle,dotstyle=o,dotsize=5pt 0,linewidth=0.8pt,arrowsize=3pt 2,arrowinset=0.25}
\begin{pspicture*}(-3.7992750623912968,-3.776781160825846)(3.7340282377430136,3.7323769364234036)
\multips(0,-3)(0,1.0){8}{\psline[linestyle=dashed,linecap=1,dash=1.5pt 1.5pt,linewidth=0.4pt,linecolor=gray]{c-c}(-3.7992750623912968,0)(3.7340282377430136,0)}
\multips(-3,0)(1.0,0){8}{\psline[linestyle=dashed,linecap=1,dash=1.5pt 1.5pt,linewidth=0.4pt,linecolor=gray]{c-c}(0,-3.776781160825846)(0,3.7323769364234036)}
\psaxes[labelFontSize=\scriptstyle,xAxis=true,yAxis=true,showorigin=false,Dx=1.,Dy=1.,ticksize=-2pt 0,subticks=0]{->}(0,0)(-3.7992750623912968,-3.776781160825846)(3.7340282377430136,3.7323769364234036)[\scriptsize{$x$},140] [\scriptsize{$f(x)$},-40]
\psplot[linewidth=1.2pt,plotpoints=200]{-3.7992750623912968}{3.7340282377430136}{-x^(2.0)+1.0}
\end{pspicture*}},				%3. Antwort rechts
				R4={\psset{xunit=0.6cm,yunit=0.6cm,algebraic=true,dimen=middle,dotstyle=o,dotsize=5pt 0,linewidth=0.8pt,arrowsize=3pt 2,arrowinset=0.25}
\begin{pspicture*}(-3.7992750623912968,-3.776781160825846)(3.7340282377430136,3.7323769364234036)
\multips(0,-3)(0,1.0){8}{\psline[linestyle=dashed,linecap=1,dash=1.5pt 1.5pt,linewidth=0.4pt,linecolor=gray]{c-c}(-3.7992750623912968,0)(3.7340282377430136,0)}
\multips(-3,0)(1.0,0){8}{\psline[linestyle=dashed,linecap=1,dash=1.5pt 1.5pt,linewidth=0.4pt,linecolor=gray]{c-c}(0,-3.776781160825846)(0,3.7323769364234036)}
\psaxes[labelFontSize=\scriptstyle,xAxis=true,yAxis=true,showorigin=false,Dx=1.,Dy=1.,ticksize=-2pt 0,subticks=0]{->}(0,0)(-3.7992750623912968,-3.776781160825846)(3.7340282377430136,3.7323769364234036)[\scriptsize{$x$},140] [\scriptsize{$f(x)$},-40]
\psplot[linewidth=1.2pt,plotpoints=200]{-3.7992750623912968}{3.7340282377430136}{-x^(-2.0)}
\end{pspicture*}},				%4. Antwort rechts
				%% LOESUNG: %%
				A1={B},				% 1. richtige Zuordnung
				A2={D},				% 2. richtige Zuordnung
				A3={C},				% 3. richtige Zuordnung
				A4={F},				% 4. richtige Zuordnung
				}
\end{beispiel}
\newpage
\pagestyle{fancy}


\begin{beispiel}[AN 1.3]{1} %PUNKTE DES BEISPIELS
			  Das Diagramm beschreibt eine Bewegung:				
				
				\begin{center}
				\psset{xunit=1.0cm,yunit=0.05cm,algebraic=true,dimen=middle,dotstyle=o,dotsize=4pt 0,linewidth=0.8pt,arrowsize=3pt 2,arrowinset=0.25}
\begin{pspicture*}(-0.8,-9)(9.192270531400977,116.1081081081061)
\multips(0,0)(0,10.0){13}{\psline[linestyle=dashed,linecap=1,dash=1.5pt 1.5pt,linewidth=0.4pt,linecolor=darkgray]{c-c}(0,0)(9.192270531400977,0)}
\multips(0,0)(0.5,0){19}{\psline[linestyle=dashed,linecap=1,dash=1.5pt 1.5pt,linewidth=0.4pt,linecolor=darkgray]{c-c}(0,0)(0,116.1081081081061)}
\psaxes[labelFontSize=\scriptstyle,xAxis=true,yAxis=true,Dx=1.,Dy=20.,ticksize=-2pt 0,subticks=2]{->}(0,0)(0.,0.)(9.192270531400977,116.1081081081061)
\psplot[linewidth=1.2pt,plotpoints=200]{0}{2}{-7.5*x^2+30*x}
\psline[linewidth=1.2pt](2.,30.)(4.5,30.)
\psplot[linewidth=1.2pt,plotpoints=200]{4.5}{6.02}{10.2536*x^2-92.9016*x+240.4229}
\psline[linewidth=1.2pt](6,52.2)(7.,80.)
\psplot[linewidth=1.2pt,plotpoints=200]{7}{9.192270531400977}{-13.7347*x^2+224.7557*x-820.2882}
\rput[tl](0.22,110){Weg (in m)}
\rput[tl](7.5,9){Zeit (in s)}
\end{pspicture*}
				\end{center}\vspace{-0,2cm}
				
				Ordne jeweils jedem Zeitintervall die entsprechende mittlere Geschwindigkeit zu!
				
				\zuordnen{
								R1={$[0;2]$},				% Response 1
								R2={$[2;4,5]$},				% Response 2
								R3={$[4,5;7]$},				% Response 3
								R4={$[7;9]$},				% Response 4
								%% Moegliche Zuordnungen: %%
								A={$0$\,m/s}, 				%Moeglichkeit A  
								B={$5$\,m/s}, 				%Moeglichkeit B  
								C={$10$\,m/s}, 				%Moeglichkeit C  
								D={$15$\,m/s}, 				%Moeglichkeit D  
								E={$20$\,m/s}, 				%Moeglichkeit E  
								F={$25$\,m/s}, 				%Moeglichkeit F  
								%% LOESUNG: %%
								A1={D},				% 1. richtige Zuordnung
								A2={A},				% 2. richtige Zuordnung
								A3={E},				% 3. richtige Zuordnung
								A4={B},				% 4. richtige Zuordnung
								}
\end{beispiel}

\begin{beispiel}[FA 1.6]{1} %PUNKTE DES BEISPIELS
Die Abbildung zeigt die Graphen der Erlös- und Kostenfunktion eines Betriebes.

Interpretiere die Koordinaten der Schnittpunkte in diesem Kontext!

\begin{center}
\psset{xunit=0.1cm,yunit=0.02cm,algebraic=true,dimen=middle,dotstyle=o,dotsize=5pt 0,linewidth=0.8pt,arrowsize=3pt 2,arrowinset=0.25}
\begin{pspicture*}(-14,-29.294042407198404)(108.54742813165333,324.82633259103704)
\multips(0,0)(0,50.0){8}{\psline[linestyle=dashed,linecap=1,dash=1.5pt 1.5pt,linewidth=0.4pt,linecolor=darkgray]{c-c}(0,0)(108.54742813165333,0)}
\multips(0,0)(10.0,0){12}{\psline[linestyle=dashed,linecap=1,dash=1.5pt 1.5pt,linewidth=0.4pt,linecolor=darkgray]{c-c}(0,0)(0,324.82633259103704)}
\psaxes[labelFontSize=\scriptstyle,xAxis=true,yAxis=false,Dx=10.,ticksize=-2pt 0,subticks=0]{->}(0,0)(0.,0)(108.54742813165333,324.82633259103704)[\scriptsize Stück,140] [,-40]
\psaxes[labelFontSize=\scriptstyle,xAxis=false,yAxis=true,Dy=50.,ylabelFactor=\,000, showorigin=false,ticksize=-2pt 0,subticks=2]{->}(0,0)(0.,0)(108.54742813165333,324.82633259103704)[,140] [\scriptsize Euro,-40]
\psplot[linewidth=0.8pt,plotpoints=200]{0}{101.5}{-0.1*x^(2.0)+10.0*x}
\psplot[linewidth=0.8pt]{0}{100}{3.030492906931785*x+58.76467703284807}
\begin{scriptsize}
\rput[tl](71.31716865751105,267.68221339816773){Kostenfunktion}
\rput[tl](85.51661645695135,140){Erlösfunktion}
\rput[tl](35,273){$E_{max}=(50\mid 250\,000)$}
\psdots[dotsize=5pt 0,dotstyle=*](0.,0.)
\psdots[dotsize=5pt 0,dotstyle=*](50.,250.)
\psdots[dotsize=5pt 0,dotstyle=*](100.,0.)
\end{scriptsize}
\end{pspicture*}
\end{center}

\antwort{Break-Even-Points:

Bei einem Absatz von 10 Stück bzw. von 60 Stück sind die Produktionskosten gleich dem Erlös (ca. 80\,000\,\euro bzw. ca. 240\,000\,\euro). Der Betrieb macht bei diesen Absatzmengen weder Gewinn noch Verlust.}
\end{beispiel}

\fancyhead{}

\fancyfoot[c]{\begin{footnotesize}Christoph Weberndorfer, Matthias Konzett\\
lama.helpme@gmail.com\end{footnotesize}}	
%\fancyfoot[c]{Christoph Weberndorfer und Matthias Konzett\\
%\hspace{0,5cm} c.weberndorfer@gmail.com und m.s.konzett@gmail.com}	
\end{document}